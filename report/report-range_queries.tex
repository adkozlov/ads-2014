\documentclass[12pt,a4paper]{article}

\usepackage[english,russian]{babel}
\usepackage[utf8]{inputenc}
\usepackage[T2A]{fontenc}
\usepackage{listings,comment,hyperref,listings,verbatim,amssymb}
\usepackage[nohead,nofoot]{geometry}
\usepackage{pgfplots}

\begin{document}

\title{Cтатическая задача в $\mathbb{R}^d$ c $O(\log^{d-1}(n))$}
\author{Андрей Козлов}
\date{Январь 2015}

\maketitle

\section*{Постановка задачи}

Требуется реализовать дерево запросов принадлежности множества точек данному прямоугольнику.

Также требуется сравнить время работы запроса на дереве со временем работы наивного алгоритма.

\section*{Реализация}

Реализованы классы \texttt{AbstractRangeTree}, \texttt{RangeTree2D} и \texttt{RangeTreeKD}, параметризуемые типом координаты точки.

Для координат реализован интерфейс \texttt{ICoordinate}. Тип координаты должен обладать следующими свойствами:
\begin{itemize}
	\item существование отношения порядка;
	\item существование следующего элемента.
\end{itemize}
\lstinputlisting[language=Java,firstline=6,frame=single,basicstyle=\footnotesize,breaklines=true]{../src/ru/spbau/ads/kozlov/rangeQuries/coordinates/ICoordinate.java}
В работе приведена реализация целых координат \texttt{IntegerCoordinate}. На основе абстрактного класса \texttt{AbstractCoordinate} можно реализовать вещественные координаты.

\section*{Проверка корректности}

Тесты на проверку корректоности проводились следующий образом:
\begin{enumerate}
	\item генерируется прямоугольник \texttt{rectangle};
	\item генерируется набор точек \texttt{pointsInside} внутри прямоугольника \texttt{rectangle};
	\item генерирутеся набор точек \texttt{pointsOutside} снаружи прямоугольника \texttt{rectangle}.
	\item наборы \texttt{pointsInside} и \texttt{pointsOutside} смешиваются и по ним строится дерево \texttt{tree};
	\item на дереве \texttt{tree} выполняется запрос на прямоугольник \texttt{rectangle}, ответ сравнивается с набором точек \texttt{pointsInside}.
\end{enumerate}

Данные тесты проводились при различных значениях числа точек снаружи и внутри прямоугоьника.

\section*{Сравнение производительности}

В таблице \ref{table:durations} приведены результаты тестирования асимптотик операций запроса. Среднее значение бралось по результатам десяти запусков.

\begin{table}[h]
	\begin{tabular}{c | c | c}
		number of points & range query & naive algorithm \\
		\hline
		10000 & 1 & 5 \\
		20000 & 0 & 0 \\
		30000 & 1 & 1 \\
		40000 & 0 & 2 \\
		50000 & 0 & 3 \\
		100000 & 2 & 9 \\
		200000 & 5 & 20 \\
		300000 & 8 & 32 \\
		400000 & 10 & 42 \\
	\end{tabular}

	\caption{Среднее время выполнения запроса (мс)}
	\label{table:durations}
\end{table}

\begin{figure}[h]
	\begin{tikzpicture}
		\begin{axis}[xlabel=number of points,ylabel=average query duration (ms)]
			\addplot[color=red,mark=x] coordinates {
				(10000, 1)
				(20000, 0)
				(30000, 1)
				(40000, 0)
				(50000, 0)
				(100000, 2)
				(200000, 5)
				(300000, 8)
				(400000, 10)
			};
			\addplot[color=blue,mark=*] coordinates {
				(10000, 5)
				(20000, 0)
				(30000, 1)
				(40000, 2)
				(50000, 3)
				(100000,9)
				(200000, 20)
				(300000, 32)
				(400000, 42)
			};
			\legend{range query,naive algorithm}
		\end{axis}
	\end{tikzpicture}

	\caption{График зависимости времени выполнения запроса от количества точек $n$}
	\label{pic:durations}
\end{figure}

\end{document}
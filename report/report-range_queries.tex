\documentclass[12pt,a4paper]{article}

\usepackage[english,russian]{babel}
\usepackage[utf8]{inputenc}
\usepackage[T2A]{fontenc}
\usepackage{listings,comment,hyperref,listings,verbatim,amssymb}
\usepackage[nohead]{geometry}
\usepackage{pgfplots}

\begin{document}

\title{Cтатическая задача в $\mathbb{R}^d$ c $O(\log^{d - 1}(n))$}
\author{Андрей Козлов}
\date{\today}

\maketitle

\section*{Постановка задачи}

Требуется реализовать дерево запросов принадлежности множества точек данному прямоугольнику с предподсчетом за $O(n \log(n))$ и запросом за $O(\log^{d - 1}(n))$, где $n$ --- количество точек.

Также требуется сравнить время работы запроса на дереве со временем работы наивного алгоритма (запрос за $O(n))$, и дерева запросов без использования ссылок (запрос за $O(\log^d(n))$).

\section*{Реализация}

Задача решалась для точек с координатами в целых числах.

Реализован абстрактный полиморфный класс \texttt{AbstractRangeTree} и его наследники \texttt{RangeTree2D} и \texttt{RangeTreeKD}, для двумерного и многомерного случаев соответсвенно.

\section*{Проверка корректности}

Тесты на проверку корректоности проводились следующий образом:
\begin{enumerate}
	\item генерируется прямоугольник \texttt{rectangle};
	\item генерируется набор точек \texttt{pointsInside} внутри прямоугольника \texttt{rectangle};
	\item генерирутеся набор точек \texttt{pointsOutside} снаружи прямоугольника \texttt{rectangle}.
	\item наборы \texttt{pointsInside} и \texttt{pointsOutside} смешиваются и по ним строится дерево \texttt{tree};
	\item на дереве \texttt{tree} выполняется запрос на прямоугольник \texttt{rectangle}, ответ сравнивается с набором точек \texttt{pointsInside}.
\end{enumerate}

Данные тесты проводились при различных значениях числа точек снаружи и внутри прямоугольника.

\section*{Сравнение производительности}

В таблице \ref{table:durations} приведены результаты тестирования асимптотик операций запроса. Среднее значение бралось по результатам тысячи запусков.

Мощности компьютера, на котором проводились испытания, не хватает для продолжения испытаний многомерного дерева запросов со ссылками. ``Выпавшее'' значение объясняется активной сборкой мусора виртуальной машиной.

Это подтверждается тем, что данный эффект повторяется  при том же значении $n$ при повторных запусках с теми же опциями виртуальной машины, а также тем, что при увеличении или уменьшении максимального размера кучи \texttt{JVM} данный эффект проявляется при меньших  или больших значениях $n$ соответственно.

Тем не менее заметен прирост производительности относительно дерева без ссылок, в чем можно убедиться на графиках \ref{pic:durations2} и \ref{pic:durations3}.

\begin{table}[h]
	\begin{tabular}{c || c | c | c || c | c | c}
		number of & $O(\log^{d - 1}(n))$, & $O(\log^d(n))$, & $O(n)$, & $O(\log^{d - 1}(n))$, & $O(\log^d(n))$, & $O(n)$, \\
		points $n$ & $d=2$ & $d=2$ & $d=2$ & $d=3$ & $d=3$ & $d=3$ \\
		\hline
		10000 & 0.21 & 0.42 & 0.26 & 0.42 & 0.18 & 0.21 \\
		20000 & 0.36 & 0.19 & 0.46 & 0.10 & 0.30 & 0.39 \\
		30000 & 0.20 & 0.34 & 0.72 & 0.15 & 0.18 & 0.80 \\
		40000 & 0.28 & 0.28 & 1.44 & 0.12 & 0.20 & 0.82 \\
		50000 & 0.31 & 0.48 & 1.16 & 0.13 & 0.28 & 1.05 \\
		100000 & 0.68 & 0.65 & 2.14 & 0.26 & 0.48 & 2.48 \\
		200000 & 1.16 & 1.64 & 5.66 & 0.51 & 0.85 & 4.88 \\
		300000 & 1.57 & 2.37 & 7.54 & 1.39 &1.83 & 7.50 \\
		400000 & 2.30 & 3.49 & 14.37 & 57.51 & 2.35 & 9.92 \\
		500000 & 3.27 & 4.52 & 14.66 & 1.40 & 2.76 & 12.48 \\
		600000 & 3.31 & 5.35 & 16.09 & & 3.20 & 16.69 \\
		700000 & 3.90 & 6.23 & 23.76 & & 3.79 & 18.41 \\
		800000 & 3.90 & 5.95 & 19.78 & & 4.78 & 26.68 \\
		900000 & 4.05 & 6.75 & 21.49 & & 5.83 & 24.46 \\
	\end{tabular}

	\caption{Среднее время выполнения запроса (мс)}
	\label{table:durations}
\end{table}

\begin{figure}[H]
	\begin{tikzpicture}
		\begin{axis}[xlabel=number of points,ylabel=average query duration (ms)]
			\addplot[color=green] coordinates {
				(10000, 0.21)
				(20000, 0.36)
				(30000, 0.20)
				(40000, 0.28)
				(50000, 0.31)
				(100000, 0.68)
				(200000, 1.16)
				(300000, 1.57)
				(400000, 2.30)
				(500000, 3.27)
				(600000, 3.31)
				(700000, 3.90)
				(800000, 3.90)
				(900000, 4.05)
			};
			\addplot[color=red] coordinates {
				(10000, 0.42)
				(20000, 0.19)
				(30000, 0.34)
				(40000, 0.28)
				(50000, 0.48)
				(100000, 0.65)
				(200000, 1.64)
				(300000, 2.37)
				(400000, 3.49)
				(500000, 4.52)
				(600000, 5.35)
				(700000, 6.23)
				(800000, 5.95)
				(900000, 6.75)
			};
			\addplot[color=blue] coordinates {
				(10000, 0.26)
				(20000, 0.46)
				(30000, 0.72)
				(40000, 1.44)
				(50000, 1.16)
				(100000, 2.14)
				(200000, 5.66)
				(300000, 7.54)
				(400000, 14.37)
				(500000, 14.66)
				(600000, 16.09)
				(700000, 23.76)
				(800000, 19.78)
				(900000, 21.49)
			};
			\legend{$O(\log^{d - 1}(n))$, $O(\log^d(n))$, $O(n)$}
		\end{axis}
	\end{tikzpicture}

	\caption{$d=2$}
	\label{pic:durations2}
\end{figure}

\begin{figure}[H]
	\begin{tikzpicture}
		\begin{axis}[xlabel=number of points,ylabel=average query duration (ms)]
			\addplot[color=green] coordinates {
				(10000, 0.42)
				(20000, 0.10)
				(30000, 0.15)
				(40000, 0.12)
				(50000, 0.13)
				(100000, 0.26)
				(200000, 0.51)
				(300000, 1.39)
				(500000, 1.40)
			};
			\addplot[color=red] coordinates {
				(10000, 0.18)
				(20000, 0.30)
				(30000, 0.18)
				(40000, 0.20)
				(50000, 0.28)
				(100000, 0.48)
				(200000, 0.85)
				(300000, 1.83)
				(400000, 2.35)
				(500000, 2.76)
				(600000, 3.20)
				(700000, 3.79)
				(800000, 4.78)
				(900000, 5.83)
			};
			\addplot[color=blue] coordinates {
				(10000, 0.21)
				(20000, 0.39)
				(30000, 0.80)
				(40000, 0.82)
				(50000, 1.05)
				(100000, 2.48)
				(200000, 4.88)
				(300000, 7.50)
				(400000, 9.92)
				(500000, 12.48)
				(600000, 16.69)
				(700000, 18.41)
				(800000, 26.68)
				(900000, 24.46)
			};
			\legend{$O(\log^{d - 1}(n))$, $O(\log^d(n))$, $O(n)$}
		\end{axis}
	\end{tikzpicture}

	\caption{$d=3$}
	\label{pic:durations3}
\end{figure}

\section*{Сравнение производительности запросов на двумерном дереве при малых значениях $K$}

$K$ --- количество точек в ответе, размерность точек в данном разделе $d = 2$.

Тесты строятся следующим образом: генерируется набор точек с координатами в диапазоне $[-10000, 10000]$, набор прямоугольников c координатами $[-100, 100]$. Считаем, что точки распределены равномерно. Таким образом, вероятность $p$ попадания точки в любой из прямоугольников не превышает $10^{-4}$.

Алгоритмы запускаются с одинаковым количеством памяти. В таблице \ref{table:durationsk} приведены полученные значения времени выполнения запроса. Соответствующие зависимости изображены на графике \ref{pic:durationsk}.

\begin{table}[h]
	\begin{tabular}{c || c | c }
		number of points $n$ & $O(\log(n))$ & $O(n)$ \\
		\hline
		10000 & 0.0052 & 0.21 \\
		25000 & 0.0048 & 0.32 \\
		50000 & 0.0052 & 0.47 \\
		75000 & 0.0060 & 0.54 \\
		100000 & 0.0046 & 1.48 \\
		250000 & 0.0074 & 1.48 \\
		500000 & 0.0118 & 2.87 \\
		750000 & 0.0116 & 4.25 \\
		1000000 & 0.0064 & 5.49 \\
	\end{tabular}

	\caption{Среднее время выполнения запроса (мс) при малых значениях $K$}
	\label{table:durationsk}
\end{table}

\begin{figure}[H]
	\begin{tikzpicture}
		\begin{axis}[xlabel=number of points,ylabel=average query duration (ms)]
			\addplot[color=green] coordinates {
				(10000,0.0052)
				(25000,0.0048)
				(50000,0.0052)
				(75000,0.0060)
				(100000,0.0046)
				(250000,0.0074)
				(500000,0.0118)
				(750000,0.0116)
				(1000000,0.0064)
			};
			\addplot[color=red] coordinates {
				(10000,0.21)
				(25000,0.32)
				(50000,0.47)
				(75000,0.54)
				(100000,0.68)
				(250000,1.48)
				(500000,2.87)
				(750000,4.25)
				(1000000,5.49)
			};
			\legend{$O(\log(n))$, $O(n)$}
		\end{axis}
	\end{tikzpicture}

	\caption{$d=2, p \le 10^{-4}$}
	\label{pic:durationsk}
\end{figure}

\end{document}
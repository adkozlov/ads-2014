\documentclass[12pt,a4paper]{article}

\usepackage[english,russian]{babel}
\usepackage[utf8]{inputenc}
\usepackage[T2A]{fontenc}
\usepackage{listings,comment,hyperref,listings,verbatim}

\begin{document}

\title{Реализация слабой кучи (weak heap).}
\author{Андрей Козлов}
\date{Январь 2015}

\maketitle

\section*{Постановка задачи}
\textit{Слабая куча} --- структура данных, позволяющая выполнять операции обычной двоичной кучи с той же ассимптотикой, но совершающая при этом меньше сравнений.

Данное свойство может быть полезно, если операции сравнения объектов являются ``дорогими'', и требуется минимизировать их количество.

Работа состоит из следующих этапов:
\begin{enumerate}
	\item реализовация двоичной и слабой куч;
	\item проверка корректности;
	\item сравнить число операций сравнения при использовании разных куч.
\end{enumerate}

\section*{Реализация}
Реализованы два параметриуемых класса \texttt{BinaryHeap} и \texttt{WeakHeap}, обладающие следующим интерфейсом:
\lstinputlisting[language=Java,firstline=6,frame=single,basicstyle=\footnotesize]{../src/ru/spbau/ads/kozlov/heaps/IHeap.java}

\section*{Тестирование}
Проверка корректности состоит из тестов на следующие операции:
\begin{enumerate}
	\item поочередное добавление случайных чисел из массива с поиском минимума на каждому шагу;
	\item построение кучи из данного массива случайных чисел с поиском минимума на каждом шагу;
	\item поочередное извлечение минимума из построенной кучи.
\end{enumerate}

Тесты проводились одновременно на обеих структурах, что позволило на каждой итерации сравнивать полученные значения минимумов друг с другом и с результатом наивного алгоритма, использующего методы класса \texttt{java.util.Collections}.

\section*{Сравнение производительности}
В качестве ``дорогой'' операции сравнения использовалась операция сравнения строк из ста символов латинского алфавита.

\begin{tabular} {r || c | c | c | c}
	heap type / \texttt{compareTo} \\ method calls count & $1e3$ & $1e4$ & $1e5$ & $1e6$ \\
	\hline
	binary & $18841$ & $255290$ & $3219373$ & $38794449$ \\
	weak & $9509$ & $128462$ & $1617733$ & $19480089$ \\
\end{tabular}

Заметим, что количество сравнений при использовании слабой кучи сокращается примерно в два раза.

\end{document}